\chapter{Transformada de Fourier en  $\mathscr{L}^2(\mathbb{R}^n)$}

\noindent Para motivar este capítulo comenzamos, haciendo alusión a las series de Fourier.  
El escenario fundamental en el que se desarrolla esta teoría era el espacio $\mathscr{L}^1(\mathbb{T})$ donde 
\begin{equation}
    \mathbb{T} = \left\{ f \in \mathscr{L}^0(\mathbb{R}), \text{ $2\pi$-periódicas e integrables en } [-\pi,\pi] \right\}.
\end{equation}

\noindent En este caso conocíamos que $\mathscr{L}^2(\mathbb{T}) \subset \mathscr{L}^1(\mathbb{T})$, y por tanto $\mathscr{L}^2(\mathbb{T})$ heredaba toda la teoría desarrollada en $\mathscr{L}^1(\mathbb{T})$. De hecho, el comportamiento de las series era especialmente agradable en este espacio.

\noindent Motivados por este hecho, surge de manera natural la imperiosa necesidad de estudiar el comportamiento de la Transformada de Fourier en $\mathscr{L}^2(\mathbb{R}^n)$. Sin embargo, nuestras expectativas se verán frustradas en cuanto consideremos la proposición \ref{prop:inco}. Esta mostraba que los espacios $\mathscr{L}^p(\mathbb{R}^n)$ eran incomparables. En particular, $\mathscr{L}^2(\mathbb{R}^n) \not\subset \mathscr{L}^1(\mathbb{R}^n)$.
\vspace{0.1cm}

\noindent En efecto, la función $f: \mathbb{R}^n \rightarrow \mathbb{R} $ definida como: 
\begin{equation}
    f(x) = \begin{cases} 
      x_1^{-1} \cdots x_n^{-1} & \text{si } x_i > 1 \, \, \forall i \in \{1,\ldots,n\}, \\
      0 & \text{en otro caso} ,
   \end{cases}
\end{equation}
verifica que $f \in \mathscr{L}^2(\mathbb{R}^n)$, pero sin embargo $f \notin \mathscr{L}^1(\mathbb{R}^n)$.

 \noindent Por tanto, el objeto $\widehat{f}$ definido en \ref{def:definicion} carece de sentido para $f$, y no podemos trasladar directamente los resultados teóricos estudiados en $\mathscr{L}^1(\mathbb{R}^n)$ a este espacio. 

\noindent Los objetivos de este capítulo son, entonces, definir el concepto de la Transformada de Fourier $\widehat{f}$ para cualquier función $f \in  \mathscr{L}^2(\mathbb{R}^n)$, estudiar sus propiedades en este espacio, y analizar la relación que guarda con la definición previamente estudiada en $\mathscr{L}^1(\mathbb{R}^n)$. Para ello, usaremos como herramienta clave el Teorema de Plancharel que enunciamos y demostramos a continuación.


\section{Definición}
Presentamos el Teorema de Plancharel, el cual pone de manifiesto que la Transformada de Fourier de una función $f \in \mathscr{L}^1(\mathbb{R}^n) \cap \mathscr{L}^2(\mathbb{R}^n) $ tiene un comportamiento especialmente agradable.
\begin{lema} \label{teoplanch}
Sea  $f \in \mathscr{L}^1(\mathbb{R}^n) \cap \mathscr{L}^2(\mathbb{R}^n) $. Entonces $\widehat{f}$ $\in$ $\mathscr{L}^2(\mathbb{R}^n)$ y se verifica que
\begin{equation}\label{ecuacion_cons}
    \int_{\mathbb{R}^n} |\widehat{f}(y)|^2 \, dy =  \int_{\mathbb{R}^n} |f(y)|^2 \, dy.
\end{equation}
Equivalentemente,
\begin{equation}
    ||f||_{L_2} =  ||\widehat{f}||_{L_2}.
\end{equation}
\end{lema}

\begin{proof}
Comenzamos observando los siguientes hechos:
\begin{itemize}
    \item  De la proposición \ref{prop:conj} deducimos que  $\overline{\widetilde{f}\, \, } \in \mathscr{L}^1(\mathbb{R}^n) \cap \mathscr{L}^2(\mathbb{R}^n)$.
    Además, por \ref{conju} sabemos que:
    \begin{equation}
\widehat{\overline{\widetilde{f}\,}}(y)= \overline{\widehat{f}(y)}\quad \forall y \in \mathbb{R}^n.
    \end{equation}

    \item Como $f,\overline{\widehat{f}\, \, }\in \mathscr{L}^1(\mathbb{R}^n) \cap \mathscr{L}^2(\mathbb{R}^n)$, se tiene que la convolución
    \begin{equation}
        g = f*\overline{\widetilde{f}\, \, }
    \end{equation}
    está definida en todo $\mathbb{R}^n$, y es uniformemente continua y acotada en $\mathbb{R}^n$. Nótese también que $g \in \mathscr{L}^1(\mathbb{R}^n)$, y por tanto podemos calcular su Transformada de Fourier
    \begin{equation}
        \widehat{g}(y) = \widehat{f}(y)\overline{\widehat{f}(y)} = |\widehat{f}(y)|^2 \quad \forall y \in \mathbb{R}^n.
    \end{equation}
    \item 
    Se verifica que
    \begin{equation}\label{eq_g}
        g(0) = (f*\overline{\widetilde{f}\,})(0) = \int_{ \mathbb{R}^n} f(y)\overline{\widetilde{f}\, \, }(-y) \, dy = \int_{\mathbb{R}^n}f(y)\overline{f(y)} \, dy = \int_{\mathbb{R}^n} |f(y)|^2 \, dy.
    \end{equation}
\end{itemize}
Nuestras expectativas de poder aplicar el Teorema de Inversión a $g$ se ven frustradas al no saber si $\widehat{g}$ está en $\mathscr{L}^1(\mathbb{R}^n)$. Es por ello por lo que recurrimos a nuestra teoría de núcleos de sumabilidad. Como $g$ es una función uniformemente continua en $\mathbb{R}^n$, usando el método de sumación de Gauss, se tiene que:

\begin{equation}
    g(x) = \lim_{\substack{\epsilon \rightarrow 0 \\ \epsilon > 0}}  \int_{\mathbb{R}^n} \widehat{g}(y)e^{2\pi i \langle x, y \rangle}  e^{-\epsilon ||y||^2} \, dy  \quad \forall x \in \mathbb{R}^n.
\end{equation}
Concretamente, para $x = 0$ se cumple 
\begin{equation}
    g(0) = \lim_{\substack{\epsilon \rightarrow 0 \\ \epsilon > 0}} \int_{\mathbb{R}^n} \widehat{g}(y)  e^{-\epsilon ||y||^2} \, dy  \quad \forall x \in \mathbb{R}^n.
\end{equation}
En particular, si tomamos una sucesión concreta que tiende a $0$:
\begin{equation}\label{convg}
        g(0) = \lim_{m \to \infty} \int_{\mathbb{R}^n} |\widehat{f}(y)|^2  e^{-\frac{1}{m} ||y||^2} \, dy  \quad \forall x \in \mathbb{R}^n.
\end{equation}
Por otro lado, aplicamos el teorema de la convergencia monótona para funciones medibles positivas a la sucesión creciente
\begin{equation}
    \{|\widehat{f}(y)|^2e^{-\frac{1}{m} ||y||^2}\},
\end{equation}

\noindent obteniendo  
\begin{equation}
    \lim_{m \to \infty} \int_{\mathbb{R}^n} |\widehat{f}(y)|^2  e^{-\frac{1}{m} ||y||^2} \, dy =   \int_{\mathbb{R}^n} \lim_{m \to \infty} |\widehat{f}(y)|^2  e^{-\frac{1}{m} ||y||^2} \, dy = \int_{\mathbb{R}^n} |\widehat{f}(y)|^2   \, dy \quad \forall y \in \mathbb{R}^n.
\end{equation}

\noindent Usando~\eqref{convg}, deducimos 
\begin{equation}
    g(0) = \int_{\mathbb{R}^n} |\widehat{f}(y)|^2   \, dy .
\end{equation}
Luego por~\eqref{eq_g} concluimos que $\widehat{f} \in \mathscr{L}^2(\mathbb{R}^n)$ y además
\begin{equation*}
    \int_{\mathbb{R}^n} |\widehat{f}(y)|^2   \, dy = \int_{\mathbb{R}^n} |f(y)|^2 \, dy. \qedhere
\end{equation*}
\end{proof}

\vspace{0.2cm}
\noindent Antes de introducir el concepto de Transformada de Fourier en $\mathscr{L}^2(\mathbb{R}^n)$, recalcamos un hecho que jugará un papel importante a lo largo de este capítulo.
\begin{proposicion}\label{propo}
    El espacio $\mathscr{L}^1(\mathbb{R}^n) \cap \mathscr{L}^2(\mathbb{R}^n)$ es un subespacio vectorial denso en $\mathscr{L}^2(\mathbb{R}^n)$.
\end{proposicion}
\begin{proof}
   La única consideración que debemos tener presente es que  por el Teorema \ref{teo:sc}, $C_{00}$ es un subespacio vectorial denso en $\mathscr{L}^2(\mathbb{R}^n)$. Y además,  $\mathscr{L}^1(\mathbb{R}^n) \cap \mathscr{L}^2(\mathbb{R}^n)$ contiene a $C_{00}(\mathbb{R}^n)$, por lo que se sigue que $\mathscr{L}^1(\mathbb{R}^n) \cap \mathscr{L}^2(\mathbb{R}^n)$ es un subespacio vectorial denso en $\mathscr{L}^2(\mathbb{R}^n)$.
\end{proof}


\begin{teorema}[Plancharel]\label{teoplan}
Existe un único operador lineal y continuo
\begin{equation}
    \mathscr{F} : L^2(\mathbb{R}^n) \rightarrow L^2(\mathbb{R}^n)
\end{equation}
tal que 
\begin{equation}\label{ecuacionop}
    \mathscr{F}(f) = \widehat{f} \quad \forall f \in L^1(\mathbb{R}^n) \cap L^2(\mathbb{R}^n).
\end{equation}
Además, el operador $\mathscr{F}$ es biyectivo y cumple la identidad
\begin{equation}
    \int_{\mathbb{R}^n} |\mathscr{F}(f)(y)|^2   \, dy = \int_{\mathbb{R}^n} |f(y)|^2 \, dy.
\end{equation}
\end{teorema}

\begin{proof}
Notemos los siguientes puntos 
\begin{itemize}
    \item El lema \ref{teoplanch} pone de manifiesto que la Transformada de Fourier define un operador lineal y continuo de $L^1(\mathbb{R}^n) \cap L^2(\mathbb{R}^n)$ en $L^2(\mathbb{R}^n)$.
    \item Por la proposición \ref{propo}, sabemos que el espacio $\mathscr{L}^1(\mathbb{R}^n) \cap \mathscr{L}^2(\mathbb{R}^n)$ es un subespacio vectorial denso en $\mathscr{L}^2(\mathbb{R}^n)$.
\end{itemize}

\noindent El teorema de extensión de operadores en espacios de Banach afirma que existe un único operador lineal y continuo $\mathscr{F}: L^2({\mathbb{R}^n}) \rightarrow  L^2({\mathbb{R}^n})$ que extiende la Transformada de Fourier en $L^1(\mathbb{R}^n) \cap L^2(\mathbb{R}^n)$, de modo que 
\begin{equation}
    \mathscr{F}(f) = \widehat{f} \quad \forall f \in L^1(\mathbb{R}^n) \cap L^2(\mathbb{R}^n).
\end{equation}

\noindent Se prueba así la primera parte del enunciado.
Proseguimos desmotrando la identidad~\eqref{ecuacionop}. Para ello, usaremos como es natural el lema \ref{teoplanch}, que prueba la identidad para funciones en $L^1(\mathbb{R}^n) \cap L^2(\mathbb{R}^n)$.

\noindent Sea $f \in  L^2(\mathbb{R}^n)$. Tomamos una sucesión $\{f_m\} \in   L^1(\mathbb{R}^n) \cap L^2(\mathbb{R}^n)$ tal que
\begin{equation}
    \lim_{m \rightarrow \infty}||f_m-f||_2=0.
\end{equation}
Se tiene entonces  
\begin{equation}
    \lim_{m \rightarrow \infty}||\mathscr{F}(f_m)-\mathscr{F}(f)||_2=0.
\end{equation}
Además, sabemos que 
\begin{equation}
    ||\mathscr{F}(f_m)||_2 =||\widehat{f}_m||_2  =||f_m||_2, 
\end{equation}
de donde deducimos lo siguiente 
\begin{equation}
    ||\mathscr{F}(f)||_2 =\lim_{m \rightarrow \infty}||\mathscr{F}(f)||_2 = \lim_{m \rightarrow \infty}||f_m||_2  =||f||_2,
\end{equation}
probando así la identidad~\eqref{ecuacionop}.
A continuación, probaremos la última aseveración del teorema.
Para ello, tenemos que comprobar que el operador $\mathscr{F}$ es inyectivo y sobreyectivo.
\begin{itemize}
    \item  Es claro que $\mathscr{F}$ es inyectivo, ya que si $f \in$  Ker($\mathscr{F}$), entonces 
    \begin{equation}
        0 = ||\mathscr{F}(f)||_2=||f||_2, 
    \end{equation}
    luego necesariamente $f=0$.
    \item Veamos que $\mathscr{F}$ es sobreyectivo. Para ello probaremos que $\mathscr{F}(L^2(\mathbb{R}^n)) = L^2(\mathbb{R}^n)$.

    Sea $f \in \mathscr{L}^1(\mathbb{R}^n) \cap \mathscr{L}^2(\mathbb{R}^n)$. Definimos para cada $m \in \mathbb{N}$, la sucesión de funciones $\{f_m\} : \mathbb{R} \rightarrow \mathbb{C}$ dada por
    \begin{equation}
        f_m(x) = \widehat{f}(-x)e^{-\frac{1}{m}||x||^2} \quad \forall x \in \mathbb{R}^n.
    \end{equation}
    Sabemos que para cada $m \in \mathbb{N}$ se tiene que $f_m \in \mathscr{L}^1(\mathbb{R}^n) \cap \mathscr{L}^2(\mathbb{R}^n)$, ya que en efecto al estar $f\in \mathscr{L}^1(\mathbb{R}^n)$, $\widehat{f}$ está acotada y el factor $e^{-\frac{1}{m}||x||^2} \in \mathscr{L}^1(\mathbb{R}^n) \cap \mathscr{L}^2(\mathbb{R}^n) $. 

    Nótese por la proposición \ref{consnuc} que
    \begin{equation}
    \begin{aligned}
        \widehat{f_m}(y) &= \int_{\mathbb{R}^n} \widehat{f}(-x)e^{-\frac{1}{m}||x||^2}e^{-2 \pi i \langle x,y \rangle} 
        = \int_{\mathbb{R}^n} \widehat{f}(x)e^{-\frac{1}{m}||x||^2}e^{2 \pi i \langle x,y \rangle} dx \\
        &= (f*K_m)(y) \quad \forall y \in \mathbb{R}, \forall m \in \mathbb{N}.
    \end{aligned}
    \end{equation}
    con $K_m(x) = (\sqrt{\pi m})^{m}G(\sqrt{\pi m }x) \quad \forall x \in \mathbb{R}, \forall n \in \mathbb{N}$.
    Luego  por el Teorema \ref{teo:suma}, se tiene que 
    \begin{equation}
       \lim_{m \rightarrow \infty} ||\mathscr{F}(f_m)-f||_2 =  \lim_{m \rightarrow \infty} ||\widehat{f_m}-f||_2 = \lim_{m \rightarrow \infty} ||f*K_m-f||_2 = 0.
    \end{equation}

    Por tanto, se tiene que $f \in \overline{\, \mathscr{F}(L^2(\mathbb{R}^n))\,}$, y como consecuencia  $L^1(\mathbb{R}^n) \cap L^2(\mathbb{R}^n) \subset \overline{\, \mathscr{F}(L^2(\mathbb{R}^n))\,}$. 
    Finalmente,
    \begin{equation}
    L^2(\mathbb{R}^n) = \overline{L^1(\mathbb{R}^n) \cap L^2(\mathbb{R}^n)} \subset \overline{\mathscr{F}(L^2(\mathbb{R}^n))}.
    \end{equation}
    donde hemos usado que $\mathscr{F}$ es una isometría de $L^2(\mathbb{R}^n)$ en $\mathscr{F}(L^2(\mathbb{R}^n))$, pudiendo así asegurar que el espacio $\mathscr{F}(L^2(\mathbb{R}^n))$ es un subespacio completo de $L^2(\mathbb{R}^n)$, y entonces cerrado en este, llegando a $(L^2(\mathbb{R}^n)) = \mathscr{F}(L^2(\mathbb{R}^n))$.
    Se tiene lo que queríamos, $\mathscr{F}(L^2(\mathbb{R}^n)) = L^2(\mathbb{R}^n)$.

    

    
\end{itemize}

Por tanto, $\mathscr{F}$ es un operador biyectivo de $L^2(\mathbb{R}^n)$ en $L^2(\mathbb{R}^n)$.
\end{proof}

\begin{observacion}
La identidad \(\mathscr{F}(f) = \widehat{f}\) para toda función \(f \in L^1(\mathbb{R}) \cap L^2(\mathbb{R})\) se debe entender en el sentido de que la clase de equivalencia de la función \(\widehat{f}\) es la misma que la de \(\mathscr{F}(f)\) . Esto significa que \(\mathscr{F}(f)(y) = f(y)\) para casi todo \(y \in \mathbb{R}^n\).
\end{observacion}
 
 \begin{observacion}
\noindent Para cada \(f \in L^2(\mathbb{R}^n)\), utilizaremos la notación \(\widehat{f}\) para referirnos a \(\mathscr{F}(f)\). Por tanto, \(\widehat{f}\) se considerará  una clase de equivalencia de \(L^2(\mathbb{R}^n)\), aunque habitualmente se piensa en \(\widehat{f}\) simplemente como una función, eligiendo un representante de su clase. En el caso particular de que \(f \in L^1(\mathbb{R}^n) \cap L^2(\mathbb{R}^n)\), entonces \(\widehat{f}\) será interpretada de manera convencional como una función en lugar de como una clase de equivalencia.

\end{observacion}

\noindent Como exploraremos en secciones futuras, hay una similitud considerable en las propiedades de la Transformada de Fourier en $\mathscr{L}^1(\mathbb{R}^n)$ y $\mathscr{L}^2(\mathbb{R}^n)$. A pesar de esto, existen algunas diferencias destacables entre ambas.

\noindent Dada una función $f \in \mathscr{L}^1(\mathbb{R}^n)$,
\begin{itemize}
    \item La clase $\widehat{f}$ puede carecer de una función representante continua;
    \item La clase $\widehat{f}$ puede carecer de una función representante acotada;
    \item La clase $\widehat{f}$ puede carecer de una función representante que cumple el lema de Riemann-Lebesgue.
\end{itemize}
\begin{ejemplo}
Sea $f: \mathbb{R}^n \rightarrow \mathbb{C}$,
\begin{equation}
    f(x) = \sum_{m=1}^{\infty}m\mathcal{X}_{]m,m+\frac{1}{2^m}[^n }(x).
\end{equation}
Se puede comprobar que $f \in \mathscr{L}^2(\mathbb{R}^n)$, por lo que es la Transformada de Fourier de alguna función en $\mathscr{L}^2(\mathbb{R}^n)$. Sin embargo, es sencillo comprobar que $f$ tiene discontinuidades de salto, por lo que no puede coincidir casi por doquier con una función continua.
\end{ejemplo}


\section{Permutación integratoria de la Transformada}
Esta propiedad era ya conocida en el espacio $\mathscr{L}^1(\mathbb{R}^n)$. Ahora examinemos cómo el resultado es análogo en $\mathscr{L}^2(\mathbb{R}^n)$.
\begin{teorema}
    Sean $f,g \in \mathscr{L}^2(\mathbb{R}^n)$. Entonces,
    \begin{equation}\label{eq:int}
        \int_{\mathbb{R}^n} \widehat{f}(x)g(x) \, dx =  \int_{\mathbb{R}^n} f(x) \widehat{g}(x) \, dx.
    \end{equation}
\end{teorema}
\begin{proof}

\noindent 

\noindent Notemos que el producto de dos funciones en $\mathscr{L}^2(\mathbb{R}^n)$ es integrable en $\mathbb{R}^n$, lo que justifica las dos integrales presentes en la ecuación~\eqref{eq:int}. 

\noindent Ahora estimaremos la siguiente cantidad.
\begin{equation}
    \left| \int_{\mathbb{R}^n}\widehat{f}(x)g(x) \, dx - \int_{\mathbb{R}^n}f(x)\widehat{g}(x) \, dx\right|.
\end{equation}

\noindent Tomamos dos sucesiones $\{f_m\},\{g_m\}$ en  $\mathscr{L}^1(\mathbb{R}^n)\cap\mathscr{L}^2(\mathbb{R}^n)$ tales que
\begin{equation}
    \lim_{m \rightarrow \infty}||f-f_m||_2 = \lim_{m \rightarrow \infty}||g-g_m||_2 =0.
\end{equation}
Como la Transformada de Fourier de una función en $\mathscr{L}^1(\mathbb{R}^n)$ es uniformemente continua en $\mathbb{R}^n$, se sigue que

\begin{equation}
    \lim_{m \rightarrow \infty}||\widehat{f}-\widehat{f}_m||_2 = \lim_{m \rightarrow \infty}||\widehat{g}-\widehat{g}_m||_2 =0.
\end{equation}
Además, también ocurre
\begin{align}
&\lim_{m \rightarrow \infty}||\widehat{f}_m||_2 = \lim_{m \rightarrow \infty}||f_m||_2 =||f||_2, \
&\lim_{m \rightarrow \infty}||\widehat{g}_m||_2 =\lim_{m \rightarrow \infty}||g_m||_2 = ||g||_2.
\end{align}
Entonces, dado $m \in \mathbb{N}$, se verifica la siguiente desigualdad
\begin{equation}
\begin{aligned}
&\left| \int_{\mathbb{R}^n}\widehat{f}(x)g(x) \, dx - \int_{\mathbb{R}^n}f(x)\widehat{g}(x) \, dx\right| \\
&= \left| \int_{\mathbb{R}^n}\left[\widehat{f}(x)g(x) -\widehat{f_m}(x)g_m(x)+\widehat{f_m}(x)g_m(x)-f(x)\widehat{g}(x) \right]\, dx \right| \\
&\leq \int_{\mathbb{R}^n}\left|[\widehat{f}(x)g(x) -\widehat{f_m}(x)g_m(x) \right| \, dx+\int_{\mathbb{R}^n}\left|f_m(x)\widehat{g_m}(x) -\widehat{f}(x)\widehat{g}(x)\right| \, dx \\
&\leq \int_{\mathbb{R}^n}\left|\widehat{(f-f_m)}(x)g(x)\right| \, dx + \int_{\mathbb{R}^n} \left|\widehat{f_m}(x)(g-g_m)(x)\right|\, dx \\
&+\int_{\mathbb{R}^n}\left|(f_m-f)(x)\widehat{g_m}(x)\right| \, dx +\int_{\mathbb{R}^n} \left|\widehat{f}(x)\widehat{(g_m-g)}(x)\right|\, dx \\ 
&\leq ||\widehat{f}-\widehat{f_m}||_2||g||_2+||\widehat{f_m}||_2||g-g_m||_2 
+||f_m-f||_2||g_m||_2+||f||_2||||\widehat{g_m}-\widehat{g}||_2.
\end{aligned}
\end{equation}

Luego tomando límite concluimos que, 
\begin{equation}
\begin{aligned}
&\left| \int_{\mathbb{R}^n}\widehat{f}(x)g(x) \, dx - \int_{\mathbb{R}^n}f(x)\widehat{g}(x) \, dx\right|  = 0,
\end{aligned}
\end{equation}

de donde se deduce la fórmula mencionada.
\end{proof}




\section{Propiedades}
A continuación, examinaremos algunas propiedades análogas a las estudiadas previamente en $\mathscr{L}^1(\mathbb{R}^n)$. De hecho, su demostración se deriva directamente de éstas.


\begin{proposicion}\label{conjugado}
Sean $f,g \in \mathscr{L}^1(\mathbb{R}^n)$. Entonces 
%\renewcommand{\labelenumi}{\roman{enumi})}

\begin{enumerate}
    \item $\widehat{\overline{f}} = \overline{\widehat{\widetilde{f}\,}\,}$,
    \item $\widehat{\widetilde{f}\,} = \widetilde{\widehat{f}}$,
    \item $\widehat{\overline{\widetilde{f}}} = \overline{\widehat{f}\,}$.
\end{enumerate}
\end{proposicion}

\begin{proof}
    \noindent Tomamos una sucesión $\{f_m\}$ en  $\mathscr{L}^1(\mathbb{R}^n)\cap\mathscr{L}^2(\mathbb{R}^n)$ tal que
\begin{equation}
    \lim_{m \rightarrow \infty}||f-f_m||_2 =0.
\end{equation}


\noindent Por la continuidad de las operaciones de conjugación $ f \mapsto \overline{f\,}$ y simetría $f \mapsto \widetilde{f}$, y usando que la Transformada de Fourier $f \mapsto \mathscr{F}(f)$ es un operador continuo en $L^2(\mathbb{R}^n)$,
\begin{equation}
    \lim_{m \rightarrow \infty}||\widehat{\overline{f}} -\widehat{\overline{f_m}}||_2 =  ||\overline{\widehat{\widetilde{f_m}\,}\,}-\overline{\widehat{\widetilde{f}\,}\,}||_2 =0.
\end{equation}

\noindent Como para cada $m \in \mathbb{N}, f_m \in \mathscr{L}^1(\mathbb{R}^n)$, podemos entonces aplicar el resultado de la Proposción \ref{conju}:
\begin{align}
||\widehat{\overline{f}} - \overline{\widehat{\widetilde{f}\,}\,}||_2 
= ||\widehat{\overline{f}} -\widehat{\overline{f_m}}-\widehat{\overline{f_m}}+\overline{\widehat{\widetilde{f}}}||_2 
\leq ||\widehat{\overline{f}} -\widehat{\overline{f_m}}||_2+||\overline{\widehat{\widetilde{f_m}\,}\,}-\overline{\widehat{\widetilde{f}\,}\,}||_2.
\end{align}
Tomando límite se tiene el primer apartado de la proposición. 

\noindent El resto de igualdades se demuestran de manera análoga, ya que lo que se ha usado realmente es la continuidad de todos los operadores involucrados y la densidad del espacio $L^1(\mathbb{R}^n) \cap L^2(\mathbb{R}^n)$ en $ L^2(\mathbb{R}^n)$, junto con la validez de las fórmulas presentadas para cualquier función de $L^1(\mathbb{R}^n) \cap L^2(\mathbb{R}^n)$.

\end{proof}
\begin{proposicion}
    Sea $f \in \mathscr{L}^2(\mathbb{R}^n)$ y sea $a \in \mathbb{R}^+$. Entonces, se tiene:
    \begin{equation}
        \widehat{H_af} = a^{-n}H_{a^{-1}}\widehat{f}.
    \end{equation}
\end{proposicion}

\begin{proof}
 \noindent Tomamos una sucesión $\{f_m\}$ en  $\mathscr{L}^1(\mathbb{R}^n)\cap\mathscr{L}^2(\mathbb{R}^n)$ tal que
\begin{equation}
    \lim_{m  \rightarrow \infty}||f-f_m||_2 =0.
\end{equation}


\noindent Por la continuidad de las operaciones de conjugación $ f \mapsto H_af\,$, y usando que $\ \widehat{}\ $ es un operador continuo en $L^2(\mathbb{R}^n)$,
\begin{equation}
    \lim_{m \rightarrow \infty}||\widehat{H_af}- \widehat{H_af_m}||_2=  \lim_{m \rightarrow \infty}||a^{-n}H_{a^{-1}}\widehat{f_m} -a^{-n}H_{a^{-1}}\widehat{f}||_2 =0.
\end{equation}

\noindent Como para cada $m \in \mathbb{N}, f_m \in \mathscr{L}^1(\mathbb{R}^n)$, podemos entonces aplicar el resultado de la Proposición \ref{esca}:
\begin{align}
||\widehat{H_af}-a^{-n}H_{a^{-1}}\widehat{f}||_2 
&= ||\widehat{H_af}- \widehat{H_af_m}+\widehat{H_af_m} -a^{-n}H_{a^{-1}}\widehat{f}||_2 \\
&\leq ||\widehat{H_af}- \widehat{H_af_m}||_2+||a^{-n}H_{a^{-1}}\widehat{f_m} -a^{-n}H_{a^{-1}}\widehat{f}||_2.
\end{align}


\noindent Tomando límite se  concluye el resultado.
\end{proof}






\begin{proposicion}\label{prop:tras}
    Sea $f \in \mathscr{L}^2(\mathbb{R}^n)$ y sea $t \in \mathbb{R}^n$. Entonces se tiene
    \begin{equation}
        \widehat{\tau_tf} =  \mu_{-t}\widehat{f}.
    \end{equation}
    
\end{proposicion}

\begin{proof}
Repetimos el argumento usado en las proposiciones anteriores. La continuidad de los operadores involucrados y la densidad del espacio $L^1(\mathbb{R}^n) \cap L^2(\mathbb{R}^n)$ en $L^2(\mathbb{R}^n)$ junto con la validez de las fórmulas presentadas para cualquier función de $L^1(\mathbb{R}^n) \cap L^2(\mathbb{R}^n)$ proporciona el resultado para funciones de  $L^2(\mathbb{R}^n)$.
\end{proof}


\begin{proposicion}
    Sea $f \in \mathscr{L}^2(\mathbb{R}^n)$ y $t \in \mathbb{R}^n$. Entonces, se tiene:
    \begin{equation*}
        \widehat{\mu_tf} = \tau_t\widehat{f}. \qedhere
    \end{equation*}
\end{proposicion}

\begin{proof}
Repetimos el argumento usado en las proposiciones anteriores. La continuidad de los operadores involucrados y la densidad del espacio $L^1(\mathbb{R}^n) \cap L^2(\mathbb{R}^n)$ en $L^2(\mathbb{R}^n)$ junto con la validez de las fórmulas presentadas para cualquier función de $L^1(\mathbb{R}^n) \cap L^2(\mathbb{R}^n)$ proporciona el resultado para funciones de  $L^2(\mathbb{R}^n)$.

\end{proof}


\section{Fórmulas de Parseval}
\begin{proposicion}\label{teo:parse}
Sean $f,g \in \mathscr{L}^2(\mathbb{R}^n)$. Entonces se cumple la identidad
\begin{equation}\label{eq:parseval}
    \int_{\mathbb{R}^n} f(x) \overline{g(x)} \, dx = \int_{\mathbb{R}^n} \widehat{f}(x) \overline{\widehat{g}(x)}\, dx.
\end{equation}
\end{proposicion}

\begin{proof}

Notemos que el producto de dos funciones en $\mathscr{L}^2(\mathbb{R}^n)$ es integrable en $\mathbb{R}^n$, lo que justifica las dos integrales presentes en la ecuación ~\eqref{eq:parseval}.
Se tiene que 
\begin{equation}\label{eq:3}
    \int_{\mathbb{R}^n} \widehat{f}(x) \overline{\widehat{g}(x)}\, dx  = \langle \widehat{f},\widehat{g} \rangle.
\end{equation}


\noindent Como $f,g \in \mathscr{L}^2(\mathbb{R}^n)$, podemos usar la identidad de polarización para la prueba.
En efecto, 
\begin{align}
    \langle \widehat{f},\widehat{g} \rangle &= \frac{1}{4} \left[ || \widehat{f}+\widehat{g}||_2^2 - || \widehat{f}-\widehat{g}||_2^2 + i|| \widehat{f}+i\widehat{g}||_2^2 - i|| \widehat{f}-i\widehat{g}||_2^2 \right] \\
    &= \frac{1}{4} \left[ || \widehat{f+g}||_2^2 - || \widehat{f-g}||_2^2 + i|| \widehat{f+ig}||_2^2 - i|| \widehat{f-ig}||_2^2 \right] \\
    &= \frac{1}{4} \left[ ||f+g||_2^2 - ||f-g||_2^2 + i|| f+ig||_2^2 - i|| f-ig||_2^2 \right] \\
    &= \langle f,g \rangle.
\end{align}

\noindent Notemos que
\begin{equation}\label{eq:2}
     \int_{\mathbb{R}^n} f(x) \overline{g(x)} \, dx = \langle f,g \rangle.
\end{equation}

\noindent   Teniendo en cuenta~\eqref{eq:2} y~\eqref{eq:3}, se concluye la prueba.
\end{proof}

\vspace{0.3cm}
\noindent Tomando $f=g$ se obtiene 
\begin{equation*}
    \int_{\mathbb{R}^n} |\widehat{f}(y)|^2   \, dy = \int_{\mathbb{R}^n} |f(y)|^2 \, dy. \qedhere
\end{equation*}
Esta igualdad, previamente establecida en \ref{ecuacion_cons}, es usada en el campo de teoría de señales.
Es común encontrar una variante de esta fórmula ajustada por un factor que depende de $2\pi$, debido a una elección distinta en la definición de la Transformada de Fourier. La fórmula muestra cómo la energía total de una señal $f$ es equivalente a la energía total de su Transformada de Fourier 
$\widehat{f}$ distribuida a través de todas sus componentes frecuenciales. Este principio, fundamental en el análisis de señales, subraya la capacidad de la Transformada de Fourier para preservar la energía de una señal a medida que se traslada del dominio del tiempo al dominio de la frecuencia. 

\begin{corolario}\label{coro}
    Sean $f,g \in \mathscr{L}^2(\mathbb{R}^n)$. Entonces se cumple la identidad
\begin{equation}
    \int_{\mathbb{R}^n} f(x) g(x) \, dx = \int_{\mathbb{R}^n} \widehat{f}(x) \widehat{g}(-x)\, dx.
\end{equation}
\end{corolario}

\begin{proof}
Usando la proposición \ref{teo:parse} se tiene 
\begin{align*}
    \int_{\mathbb{R}^n} f(x) g(x) \, dx = \int_{\mathbb{R}^n} f(x) \overline{\overline{g}(x)} \, dx
    = \int_{\mathbb{R}^n} \widehat{f}(x) \overline{\widehat{\overline{g}}(x)} \, dx.
\end{align*}
Finalmente, haciendo uso del primer apartado de la proposición, \ref{conjugado}
\begin{align*}
     \int_{\mathbb{R}^n} \widehat{f}(x) \overline{\widehat{\overline{g}}(x)} \, dx = \int_{\mathbb{R}^n} \widehat{f}(x) \overline{\overline{\widehat{g}(-x)}} \, dx = \int_{\mathbb{R}^n} \widehat{f}(x) \widehat{g}(-x) \, dx,
\end{align*}
concluyendo lo que se pedía.
\end{proof}


\section{Convolución}

En este contexto teórico, introducimos el Teorema de Convolución, cuya conclusión coincide con el que ya hemos demostrado previamente para $\mathscr{L}^1(\mathbb{R}^n)$.

\begin{teorema} (Teorema de Convolución). Sean $f \in \mathscr{L}^1(\mathbb{R}^n), g \in \mathscr{L}^2(\mathbb{R}^n)$. Entonces se tiene que 
\begin{equation}
    \widehat{f*g} = \widehat{f} \,\widehat{g}.
\end{equation}
    
\end{teorema}

\begin{proof}    
Naturalmente, nuestro objetivo es aplicar el resultado obtenido para funciones en $\mathscr{L}^1(\mathbb{R}^n)$. En efecto, $f \in \mathscr{L}^1(\mathbb{R}^n)$, sin embargo, $g \not\in \mathscr{L}^1(\mathbb{R}^n)$, pero esto no es problema ya que podemos razonar como hemos hecho anteriormente.
Tomamos una sucesión de funciones $\{g_m\}$ en $\mathscr{L}^1(\mathbb{R}^n) \cap \mathscr{L}^2(\mathbb{R}^n)$ de modo que 
\begin{equation}
  \lim_{m \rightarrow \infty}||g-g_m||_2 = 0.   
\end{equation}

\noindent Como $f \in \mathscr{L}^1(\mathbb{R}^n)$ y para cada $m \in \mathbb{N}$, $g_m \in \mathscr{L}^1(\mathbb{R}^n)$, podemos entonces aplicar el Teorema \ref{teo}, obteniendo que
\begin{equation}
\widehat{f*g_m} = \widehat{f} \, \,\widehat{g_m} \quad \forall m \in \mathbb{N}.
\end{equation}

\noindent En consecuencia, para cada $m \in \mathbb{N}$,

\begin{align}
||\widehat{fg}-\widehat{f}  \, \widehat{g}||_2 &= ||\widehat{fg}-\widehat{fg_m} + \widehat{fg_m}-\widehat{f} \widehat{g}||_2 \\
&\leq ||\widehat{fg}-\widehat{fg_m}||_2 + ||\widehat{fg_m}-\widehat{f} \widehat{g}||_2 \\
&= ||(f*(g-g_m))\,\,\widehat{}\,\,||_2 + ||\widehat{f}(\widehat{g_m}-\widehat{g})||_2 \\
&\leq||(f*(g-g_m))||_2 + ||\widehat{f}||_{\infty}||g-g_m||_2 
\leq
 2||f||_1||g-g_m||_2.
\end{align}

\noindent Tomando límite se tiene que
\begin{equation}
    ||\widehat{fg}-\widehat{f}  \, \widehat{g}||_2  \leq \lim_{m \rightarrow \infty}2||f||_1||g-g_m||_2  = 0,
\end{equation}
lo cual demuestra $\widehat{f*g} = \widehat{f} \,\widehat{g}$.
\end{proof}

\begin{observacion}
El lector podría sorprenderse al observar que, a diferencia de la mayoría de los enunciados presentados en este capítulo, en este caso las dos funciones no pertenecen a $\mathscr{L}^2(\mathbb{R}^n)$. Esto se debe a que en el caso de que ambas funciones estén en $\mathscr{L}^2(\mathbb{R}^n)$, la convolución de estas no tiene por qué estar en $\mathscr{L}^2(\mathbb{R}^n)$ necesariamente. Es por ello que imponemos que una esté  en $\mathscr{L}^1(\mathbb{R}^n)$, en cuyo caso sí se tiene por el Teorema \ref{teo:conv2}.
\end{observacion}


\noindent Como ya adelantábamos en capítulos anteriores, el siguiente resultado no tenía sentido en $\mathscr{L}^1(\mathbb{R}^n)$. Sin embargo, el producto de funciones en  $\mathscr{L}^2(\mathbb{R}^n)$ es una función integrable en $\mathbb{R}^n$, lo cual permite presentar el siguiente resultado.

\begin{teorema}Sean $f,g \in \mathscr{L}^2(\mathbb{R}^n)$. Entonces se tiene que 
\begin{equation}
    \widehat{fg} = \widehat{f}*\widehat{g}.
\end{equation}
\end{teorema}

\begin{proof}
Dado $y \in \mathbb{R}^n$, usando el Corolario \ref{coro}, se obtiene que 
\begin{equation*}
\begin{aligned}
\widehat{(fg)}(y) & = \int_{\mathbb{R}^n} f(x)g(x) e^{-2 \pi i \langle x, y \rangle} \, dx 
 = \int_{\mathbb{R}^n} f(x)(\mu_{-y}g)(x) \, dx = \int_{\mathbb{R}^n} \widehat{f}(t)(\widehat{\mu_{-y}g})(-t) \, dt \\
& = \int_{\mathbb{R}^n} \widehat{f}(t)\widehat{g}(y-t)\,dt = (\widehat{f}*\widehat{g})(y).
\end{aligned}\qedhere
\end{equation*}
\end{proof}

\vspace{0.2cm}
\noindent 
De este modo, la convolución en el dominio temporal se convierte en multiplicación en el dominio de la frecuencia, mientras que la operación de multiplicar en el dominio temporal se traduce en realizar la convolución en el dominio de la frecuencia. 


\noindent Una posible aplicación de este último resultado en el ámbito computacional es el diseño de filtros complejos a partir de la combinación de filtros más simples. Al multiplicar dos señales en el tiempo y tomar la Transformada de Fourier del producto, se puede obtener el efecto equivalente a convolucionar sus espectros, lo que es útil para la creación de respuestas de filtro específicas.




\section{Teorema de Inversión}

Probaremos a continuación el Teorema de Inversión. Como ya estudiamos en capítulos anteriores, este proporciona una manera de recuperar la función a partir de la Transformada de Fourier de esta. Sin embargo, en este marco teórico no hacen falta hipótesis adicionales como ocurría en  $\mathscr{L}^1(\mathbb{R}^n)$.

\begin{teorema}
    Sea $f \in \mathscr{L}^2(\mathbb{R}^n)$. Entonces
    \begin{equation}
        f(x) = \widehat{\widehat{f}(}-x) \, \, c.p.d \,\, \text{en} \,\,  \mathbb{R}^n.
    \end{equation}
\end{teorema}


\begin{proof}

\noindent Dada $g \in \mathscr{L}^2(\mathbb{R}^n)$. Usando la proposición \ref{teo:parse} se tiene 
\begin{equation}
    \int_{\mathbb{R}^n}\widetilde{f}(x)\overline{g(x)} \, dx =\int_{\mathbb{R}^n}\widehat{\widetilde{f}(}x)\overline{\widehat{g}(x)} \, dx = \int_{\mathbb{R}^n}\widehat{\widetilde{f}(}-x)\overline{\widehat{g}(x)} \, dx =\int_{\mathbb{R}^n}\widehat{f(}x)\overline{\widehat{g}(-x)} \, dx.
\end{equation}
Por el Corolario \ref{coro}, sabemos que 
\begin{equation}
\int_{\mathbb{R}^n}\widehat{f(}x)\overline{\widehat{g}(-x)} \, dx  = \int_{\mathbb{R}^n}\widehat{\widehat{f(}}x)\overline{g(x)} \, dx.
\end{equation}
Luego, 
\begin{equation}
    \int_{\mathbb{R}^n}\widetilde{f}(x)\overline{g(x)} \, dx  =\int_{\mathbb{R}^n}\widehat{\widehat{f(}}x)\overline{g(x)} \, dx \quad \forall g \in \mathscr{L}^2(\mathbb{R}^n).
\end{equation}

\noindent De donde deducimos que $\widetilde{f} = \widehat{\widehat{f}\,}$, y por tanto $f(x) = \widehat{\widehat{f}(}-x) \, \, c.p.d \,\, \text{en} \,\,  \mathbb{R}^n$.
\end{proof}


