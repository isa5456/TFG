% !TeX root = ../tfg.tex
% !TeX encoding = utf8

\chapter{Transformada de Fourier Discreta}

\section{Motivación}

En este capítulo, deduciremos la Transformada Discreta de Fourier (DFT), una herramienta crucial en el procesamiento digital de señales.
Existen diversas aproximaciones que derivan en la expresión de la DFT. 

Esta definición podría darse directamente y luego tratar de enlazarla con el concepto que teníamos para funciones integrables en $\mathbb{R}$, sin embargo, se opta por tratar de desarrollar la intuición de la definicón de esta partiendo  del ámbito continuoya que proporcionar una conexión conceptual con el dominio continuo contribuye a una comprensión más profunda y coherente de la transformada discreta de Fourier.

En cierta manera el lector puede pensar que trataremos de "transformar lo continuo en discreto".


\section{1-DFT}

Sea $f : \mathbb{R} \rightarrow  \mathbb{R}$  una función integrable en $\mathbb{R}$ que representará una señal. Haremos dos suposiciones acerca de esta.
\begin{itemize}
    \item Suponemos que \( f \) es de soporte compacto en \([0, L]\), donde \( L \in \mathbb{N} \).
    \item Suponemos que \( f^* \) es de soporte compacto en  \([0, 2B]\), donde \( B \in \mathbb{N} \).
\end{itemize}


\noindent Estamos considerando  por tanto $f$ tanto limitada en el tiempo como limitada en banda, \textbf{sabiendo que esto solo puede ser aproximadamente cierto}. Sin embargo, finalmente vamos a llegar a una definición de una transformada discreta de Fourier que tendrá sentido por sí misma, independientemente de estas precarias suposiciones iniciales.

\vspace{0.9mm}
\underline{\textit{Paso1}}

Tomamos $N$ muestras equiespaciadas de $f$  en $[0,L]$ con \[ N = \frac{L}{\frac{1}{2B}} = 2BL \].

Obteniendo  los puntos
\[ t_0 = 0, \quad t_1 = \frac{1}{2B}, \quad t_2 = \frac{2}{2B}, \ldots, t_{N-1} = \frac{N-1}{2B} \].

Consideramos la versión discreta de \( f \) como la lista \( [f(t_0), f(t_1), \ldots, f(t_{N-1})] \). 






En el contexto del \textbf{Teorema de Muestreo} es una buena elección como tasa de muestreo:

\vspace{1.5cm}


\noindent \underline{\textit{Paso2}}

\noindent Sea \( g: \mathbb{R} \rightarrow \mathbb{R} \) definida como 
\begin{equation}
    g(t) = \sum_{n=0}^{N-1} \delta(t - t_n) \cdot f(t_n)
\end{equation} 
 es la versión discreta de $f$ definida en $\mathbb{R}$ 

Construimos ahora $\widehat{g}$ como: 

\begin{equation}
    \widehat{g}(s) = \sum_{n=0}^{N-1} f(t_n)  \cdot \widehat{\delta}(t - t_n) = \sum_{n=0}^{N-1} f(t_n)  \cdot  e^{-2\pi i s t_n}
\end{equation}

\underline{\textit{Paso3}}
\vspace{1.0mm}

\textbf{Necesito justificar transformada de g}

Discretizamos $\widehat{g}$

Tomamos $N$ muestras equiespaciadas de $f$  $[0,2B]$ con \[ N = \frac{2B}{\frac{1}{L}} = 2BL \].

Obteniendo  los puntos
\[ s_0 = 0, \quad s_1 = \frac{1}{L}, \quad s_2 = \frac{2}{L}, \ldots, s_{N-1} = \frac{N-1}{L} \].

Consideramos la versión discreta de \( \widehat{g} \) como la lista \( [\widehat{g} (s_0), \widehat{g} (s_1), \ldots, \widehat{g} (s_{N-1})] \). 

Obteniendo finalmente:

\begin{equation}
    \widehat{g}(s_n) = \sum_{k=0}^{N-1} f(t_k) e^{-2\pi i s_n t_k}, \quad n = 0,1,\ldots,N-1
\end{equation}



\underline{\textit{Paso4}}


\[
Ff(s) = \int_{0}^{L} e^{-2\pi ist} f(t) \, dt.
\]
Así, en los puntos de muestra $s_m$,
\[
Ff(s_m) = \int_{0}^{L} e^{-2\pi ism t} f(t) \, dt,
\]
y conocer los valores de $Ff(s_m)$ es conocer $Ff(s)$ razonablemente bien. Ahora usa los puntos de muestra $t_k$ para $f(t)$ y escribe una aproximación de suma de Riemann para la integral. El espaciado $\Delta t$ de los puntos es $1/2B$, entonces
\[
Ff(s_m) = \int_{0}^{L} f(t)e^{-2\pi ism t} \, dt \approx \frac{1}{2B} \sum_{n=0}^{N-1} f(t_n) e^{-2\pi ism t_n} \Delta t = \frac{1}{2B} \sum_{n=0}^{N-1} f(t_n) e^{-2\pi ism t_n} = \frac{1}{2B} F(s_m).
\]


















\vspace{1.5cm}

\begin{definicion}   
Sea $f = (f[0],f[1],\ldots,f[N-1])$ una N-tupla. Definimos la Transformada de Fourier Discreta de $f$ como la N-tupla $F = (F[0],F[1],\ldots,F[N-1])$ definida como:
\begin{equation}
    F[m] = \sum_{n=0}^{N-1} f[n] e^{-2\pi i \frac{mn}{N}}, \quad m = 0,1,\ldots,N-1
\end{equation}
\end{definicion}


\noindent De donde se sigue inmediatamente la siguiente definición.

\begin{definicion}   
Sea $F = (F[0],F[1],\ldots,F[N-1])$ una N-tupla. Definimos la Inversa de la Transformada de Fourier Discreta de $F$ como la N-tupla $f = (f[0],f[1],\ldots,f[N-1])$ definida como:
\begin{equation}
    f[n] = \frac{1}{N}\sum_{m=0}^{N-1} F[m] e^{2\pi i \frac{mn}{N}}, \quad n = 0,1,\ldots,N-1
\end{equation}
\end{definicion}

\noindent Recibe el nombre de Inversa de la Transformada de Fourier Discreta ya que a partir de una comprobación directa obtenemos la identidad al componer ambas. \textbf{(hacer una dirección )}
\textbf{escala}


























\endinput
%--------------------------------------------------------------------
% FIN DEL CAPÍTULO. 
%--------------------------------------------------------------------
