% !TeX root = ../tfg.tex
% !TeX encoding = utf8
%
%*******************************************************
% Introducción
%*******************************************************

% \manualmark
% \markboth{\textsc{Introducción}}{\textsc{Introducción}} 

\chapter{Introducción}

\noindent Al aplicar la transformada de Fourier a una señal en el dominio del tiempo, obtenemos su representación en el dominio de la frecuencia. Esto nos permite examinar la señal original como una composición en términos de componentes sinusoidales.

\noindent Una imagen no es más que una señal visual, y es así susceptible de ser representada en el dominio de la frecuencia. Allí, trabajando con la magnitud y la fase, no solo podremos recuperar la imagen original, sino que ciertas tareas como extracción de características, compresión o  eliminación de ruido, se verán significativamente simplificadas.

\noindent Dado que partimos de datos discretos, y que los ordenadores solo pueden manejar sumas finitas, surge la  imperiosa necesidad de definir en este ambiente una 'aproximación' de la transformada de Fourier.


\noindent \textbf{Completar al final del trabajo.}


\endinput
