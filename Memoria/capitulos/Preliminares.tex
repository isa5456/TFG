\chapter{Preliminares}

De ahora en adelante trabajaremos con el espacio vectorial $\mathbb{R}^n$. En este espacio definimos el producto escalar de dos vectores \( x = (x_1, x_2, \ldots, x_n) \in \mathbb{R}^n \) e \( y = (y_1, y_2, \ldots, y_n) \in \mathbb{R}^n \) como el número real $\langle x, y \rangle$ dado por

\begin{equation}\label{eq:prod}
    \langle x, y \rangle = \sum_{k=1}^{n} x_k y_k.
\end{equation}
 
\noindent El espacio $\mathbb{R}^n$ dotado del producto escalar definido en~\eqref{eq:prod} es un espacio de Hilbert, que se conoce como el espacio euclídeo $n$-dimensional. 

\noindent La norma asociada a
dicho producto escalar es la norma euclídea $||\cdot||_2$. El espacio euclídeo $n$-dimensional es un espacio normado con esta norma. En $\mathbb{R}^n$ es posible considerar otras normas distintas. Sin embargo, en ausencia de especificación, se presume el uso de la norma euclídea.

\noindent El espacio $\mathbb{R}^n$ es además un espacio topológico con la topología usual que es la generada por la distancia asociada a la norma en $\mathbb{R}^n$. Es indiferente que norma se considere ya que todas ellas son equivalentes.


\section{Espacios $\mathscr{L}^p(\mathbb{R}^n)$}

\noindent Sea $\mathscr{L}^0(\mathbb{R}^n) = \{f: \mathbb{R}^n \rightarrow \mathbb{C}\ : \  f \ \text{es medible}\}$. Para cada $p \in \mathbb{R}$ con $p \geq 1$, se define
\begin{equation}
    \mathscr{L}^p(\mathbb{R}^n) = \left\{ f \in \mathscr{L}^0(\mathbb{R}^n): \int_{\mathbb{R}^n} |f(x)|^p \, dx < \infty \right\},
\end{equation}

\noindent y para cada función $f \in \mathscr{L}^p(\mathbb{R}^n)$, se define la seminorma
\begin{equation}
    ||f||_p = \left( \int_{\mathbb{R}^n}|f(x)|^p  \, dx \right)^{\frac{1}{p}}.
\end{equation}

\noindent Definimos también 
\begin{equation}
    \mathscr{L}^{\infty}(\mathbb{R}^n) = \{f \in \mathscr{L}^{0}(\mathbb{R}^n): f \, \text{está esencialmente acotada en}\, \mathbb{R}^n\},
\end{equation}

\noindent y, dada una función $f \in \mathscr{L}^\infty(\mathbb{R}^n)$, se define
\begin{equation}
    ||f||_\infty = \inf\{M \in [0, \infty[\,: |f(x)|\leq M \, \text{c.p.d. en}\, \mathbb{R}^n\}.
\end{equation}

\noindent Para cada $1 \leq p \leq \infty$, el espacio $L^p(\mathbb{R}^n)$ es el espacio de Banach originado por la identificación en $\mathscr{L}^p(\mathbb{R}^n)$ de las funciones que coinciden casi por doquier.
Por lo que los elementos de $L^p(\mathbb{R}^n)$ no son funciones sino clases de funciones bajo la relación
de equivalencia de “ser iguales casi por doquier”. 

\noindent El espacio \( L^2(\mathbb{R}^n) \) es digno de atención especial, siendo un espacio de Hilbert cuyo producto escalar se define mediante la fórmula
\begin{equation}
    \langle f,g \rangle = \int_{\mathbb{R}^n}f(x)\overline{g(x)}\, dx \quad \forall f,g \in  L^2(\mathbb{R}^n).
\end{equation}

\begin{proposicion}
    Sea \(f,g \in L^2(\mathbb{R}^n)\). Se verifica la  identidad
    \begin{equation}
         \langle f,g \rangle = \frac{1}{4} \left[ ||f+g||_2^2 - ||f-g||_2^2 + i||f+ig||_2^2 - i||f-ig||_2^2 \right].
    \end{equation}
denominada identidad de polarización.
\end{proposicion}


\begin{proposicion} \label{prop:inco}
Los espacios $\mathscr{L}^p(\mathbb{R}^n)$ son mutuamente incomparables. 
\end{proposicion}
\begin{proof}
    Sea $1 \leq p < q \leq \infty$. Fijamos $p < a < q$ y definimos la siguiente función $f: \mathbb{R}^n \rightarrow \mathbb{R}$ como
    \begin{equation}
        f(x) = \begin{cases}
    \prod_{i=1}^n x_i^{-1/a} & \text{si } x_i > 1 \, \, \forall i \in \{1,\ldots, n\}, \\
    0 & \text{en otro caso. } 
\end{cases}
    \end{equation}
Entonces se tiene que $f \notin \mathscr{L}^p(\mathbb{R}^n)$ y $f \in \mathscr{L}^q(\mathbb{R}^n)$.

\noindent Por otro lado definimos $g: \mathbb{R}^n \rightarrow \mathbb{R}$  como
 \begin{equation}
        f(x) = \begin{cases}
    \prod_{i=1}^n x_i^{-1/a} & \text{si } 0< x_i < 1 \, \, \forall i \in \{1,\ldots, n\}, \\
    0 & \text{en otro caso. } 
\end{cases}
\end{equation}

\noindent Es claro que $f \notin \mathscr{L}^q(\mathbb{R}^n)$ y $f \in \mathscr{L}^p(\mathbb{R}^n)$.

\noindent Concluimos entonces que   los conjuntos $\mathscr{L}^q(\mathbb{R}^n)$ y $\mathscr{L}^p(\mathbb{R}^n)$ son incomparables:
$\mathscr{L}^q(\mathbb{R}^n) \not\subset \mathscr{L}^p(\mathbb{R}^n)$ y $\mathscr{L}^p(\mathbb{R}^n) \not\subset \mathscr{L}^q(\mathbb{R}^n)$.
\end{proof}

\begin{proposicion}
Sea $f\in \mathscr{L}^1(\mathbb{R}^n)$. Entonces, para cada $t \in \mathbb{R}^n$, la función $x \mapsto f(x-t)$ es integrable en $\mathbb{R}^n$ y se cumple la identidad
\begin{equation}
    \int_{\mathbb{R}^n}f(x-t) \, dx = \int_{\mathbb{R}^n}f(x)\, dx.
\end{equation}
\end{proposicion}


\noindent Definimos otros dos espacios con los que también trabajaremos.
\begin{definicion}
    El conjunto $C_{0}(\mathbb{R}^n)$ es el conjunto de las funciones continuas $f: \mathbb{R}^n \rightarrow \mathbb{C}$ que se anulan en infinito.
\end{definicion}
\begin{definicion}
    El conjunto $C_{00}(\mathbb{R}^n)$ es el conjunto de las funciones continuas $f: \mathbb{R}^n \rightarrow \mathbb{C}$ cuyo soporte es compacto.
\end{definicion}
\begin{proposicion}
    $C_{0}(\mathbb{R}^n)$ es un espacio normado con la norma 
    \begin{equation}
        ||f||_\infty = \max\{|f(x)|: x \in \mathbb{R}^n\} ,\quad \forall f \in C_{0}(\mathbb{R}^n).
    \end{equation}
    y se tiene que  $C_{00}(\mathbb{R}^n)$ es un subespacio denso en $C_{0}(\mathbb{R}^n)$.
\end{proposicion}


\noindent Para $ 1 \leq p \leq \infty$, se tiene que $C_{00}(\mathbb{R}^n) \subset \mathscr{L}^p(\mathbb{R}^n) $. De hecho, sabemos que $C_{00}(\mathbb{R}^n)$ es denso en $\mathscr{L}^p(\mathbb{R}^n)$. 
\begin{teorema}\label{teo:sc}
    Sea $1 \leq p < \infty$ y sea $f \in \mathscr{L}^p(\mathbb{R}^n)$. Entonces para cada $\epsilon \in \mathbb{R}^+$, existe $g \in C_{00}(\mathbb{R}^n)$ tal que $||f-g||_p < \epsilon$.
\end{teorema}

\section{Definiciones}

 \begin{definicion}
    Sea $f \in \mathscr{L}^0(\mathbb{R}^n)$, para cada $a \in \mathbb{R}^+$, definimos $H_af \in \mathscr{L}^0(\mathbb{R}^n) $ como $H_af(x)=f(ax).$
\end{definicion}
\begin{proposicion}
    Para $ 1\leq p \leq \infty$, sea $f \in \mathscr{L}^p(\mathbb{R}^n)$. Entonces, para cada $a \in \mathbb{R}^+$,  $H_af \in \mathscr{L}^p(\mathbb{R}^n) $ y $||H_af||_p = a^{-n/p}||f||_p$.
\end{proposicion}

\begin{proof}
Sea $ 1\leq p < \infty$, y $a \in \mathbb{R}^+$. Usaremos que $f \in \mathscr{L}^p(\mathbb{R}^n)$. Entonces realizando un cambio de variable adecuado se tiene que
\begin{equation}
       \int_{\mathbb{R}^n} |H_af(x)|^p  \, dx =\int_{\mathbb{R}^n} |f(ax)|^p  \, dx =  a^{-n}\int_{\mathbb{R}^n} |f(x)|^p  \, dx = a^{-n}||f||_p^p< \infty.
\end{equation}
    
\noindent De donde se deduce que $||H_af||_p = a^{-n/p}||f||_p$.
\vspace{0.2cm}

\noindent Para $p= \infty$, se tiene que $f$ está esencialmente acotada en $\mathbb{R}^n$. Entonces, 
  como $|H_af(x)|^p= |f(ax)|  \,\, \forall x \in \mathbb{R}^n$, se tiene que $\tau_t{f}$ está esencialmente acotada en $\mathbb{R}^n$ y  el supremo esencial coincide evidentemente en ambos casos.
\end{proof}




\begin{definicion}
Sea $f \in \mathscr{L}^0(\mathbb{R}^n)$. Para cada $t \in \mathbb{R}^n$, definimos 
\begin{itemize}
    \item $\tau_tf \in \mathscr{L}^0(\mathbb{R}^n) $ como $\tau_tf(x)=f(x-t) \,\,\forall x \in \mathbb{R}^n,$
    \item $\mu_tf \in \mathscr{L}^0(\mathbb{R}^n) $ como $\mu_tf(x)=e^{2 \pi i  \langle x, t \rangle}f(x) \,\, \forall x \in \mathbb{R}^n$.
\end{itemize}
\end{definicion}

\begin{proposicion}
    Para $ 1\leq p \leq \infty$, sea $f \in \mathscr{L}^p(\mathbb{R}^n)$. Entonces, para cada  $t \in \mathbb{R}^n$,  $\tau_tf,\mu_tf \in \mathscr{L}^p(\mathbb{R}^n) $ y $||\tau_tf||_p = ||\mu_tf||_p = ||f||_p$.
\end{proposicion}
\begin{proof}
Sea $ 1\leq p < \infty$, y $t \in \mathbb{R}^n$. Entonces
    \begin{equation}
       \int_{\mathbb{R}^n} |\tau_tf(x)|^p  \, dx =\int_{\mathbb{R}^n} |f(x-t)|^p  \, dx =  \int_{\mathbb{R}^n} |f(x)|^p  \, dx = ||f||_p^p< \infty,
    \end{equation}
    \begin{equation}
        \int_{\mathbb{R}^n} |\mu_tf(x)(x)|^p  \, dx = \int_{\mathbb{R}^n} |e^{2 \pi i  \langle x, t \rangle}f(x)|^p  \, dx= \int_{\mathbb{R}^n} |f(x)|^p  \, dx = ||f||_p^p< \infty.
    \end{equation}
De donde se deduce que $||\tau_tf||_p = ||\mu_tf||_p = ||f||_p$.
\vspace{0.2cm}

\noindent Para $p= \infty$, se tiene que $f$ está esencialmente acotada en $\mathbb{R}^n$. Entonces, 
\begin{itemize}
    \item   Como $|\tau_t{f}(x)| = |f(x-t)| = |f(x)| \quad \forall x \in \mathbb{R}^n$, se tiene que $\tau_t{f}$ está esencialmente acotada en $\mathbb{R}^n$.
    \item Como $|\mu_t(x)| = |e^{2 \pi i  \langle x, t \rangle}f(x)|= |f(x)|  \quad \forall x \in \mathbb{R}^n$, se sigue que $\mu_t{f}$ está esencialmente acotada en $\mathbb{R}^n$.
\end{itemize} 
Además, el supremo esencial coincide evidentemente en los tres casos.
\end{proof}

\begin{definicion}
Sea $f \in \mathscr{L}^0(\mathbb{R}^n)$, definimos 
\begin{itemize}
    \item $\overline{f} \in \mathscr{L}^0(\mathbb{R}^n) $ como $\overline{f}(x)=\overline{f(x)} \quad \forall x \in \mathbb{R}^n,$
    \item $\widetilde{f} \in \mathscr{L}^0(\mathbb{R}^n) $ como $\widetilde{f}(x)=f(-x) \quad \forall x \in \mathbb{R}^n.$
\end{itemize}
\end{definicion}

\begin{proposicion} \label{prop:conj}
    Para $ 1\leq p \leq \infty$, sea $f \in \mathscr{L}^p(\mathbb{R}^n)$. Entonces,   $\overline{f},\widetilde{f} \in \mathscr{L}^p(\mathbb{R}^n) $, y $||\overline{f}||_p = ||\widetilde{f}||_p = ||f||_p$.
\end{proposicion}

\begin{proof}
Sea $ 1\leq p < \infty$. Entonces
    \begin{equation}
       \int_{\mathbb{R}^n} |\overline{f}(x)|^p  \, dx =  \int_{\mathbb{R}^n} |\overline{f(x)}|^p  \, dx = \overline{\int_{\mathbb{R}^n} |f(x)|^p  \, dx} = ||f||_p^p< \infty,
    \end{equation}
    \begin{equation}
        \int_{\mathbb{R}^n} |\widetilde{f}(x)|^p  \, dx = \int_{\mathbb{R}^n} |f(-x)|^p  \, dx= ||f||_p^p< \infty.
    \end{equation}
De donde se deduce que $||\overline{f}||_p = ||\widetilde{f}||_p = ||f||_p$.
\vspace{0.2cm}

\noindent Para $p= \infty$, se tiene que $f$ está esencialmente acotada en $\mathbb{R}^n$. Entonces, 
\begin{itemize}
    \item   Como $|\widetilde{f}(x)| = |f(-x)| \,\, \forall x \in \mathbb{R}^n$, se tiene que $\widetilde{f}$ está esencialmente acotada en $\mathbb{R}^n$.
    \item Como $|\overline{f}(x)| = |\overline{f(x)}|= |f(x)|  \,\, \forall x \in \mathbb{R}^n$, se sigue que $\overline{f}$ está esencialmente acotada en $\mathbb{R}^n$.
\end{itemize} 
Además, el supremo esencial coincide evidentemente en los tres casos.
\end{proof}

\noindent Estos resultados se usarán en adelante sin hacer mención explícita a ellos. 

\section{Módulo de Continuidad}
\begin{definicion}
Sea $1\leq p < \infty$ y sea $f \in \mathscr{L}^p(\mathbb{R}^n)$. El \textit{módulo de continuidad en media} de $f$ es la función $w_pf: \mathbb(\mathbb{R}^n) \rightarrow \mathbb{R}$ definida por 
\begin{equation}
    w_pf(t) =||\tau_tf-f||_p  \quad \forall t \in \mathbb{R}^n.
\end{equation}
\end{definicion}

\begin{proposicion}\label{mod_cont}
Sea $1\leq p < \infty$ y sea $f \in \mathscr{L}^p(\mathbb{R}^n)$. Entonces, 
\begin{itemize}
    \item $w_pf(t)=w_pf(-t) \quad \forall t \in \mathbb{R}^n$,
    \item $0 \leq w_pf(t)  \leq 2||f||_p \quad \forall t \in \mathbb{R}^n$,
    \item $w_pf$ es uniformemente continua en $\mathbb{R}^n$. En particular, $\underset{\substack{t \rightarrow 0}}{\lim}w_pf(t)=0.$
\end{itemize}
\end{proposicion}
\begin{proof}
Comenzamos probando la simetría de $w_pf$. Para ello, tomamos $t \in \mathbb{R}^n$, y observamos que:
\begin{equation}
    w_pf(t) = \left(\int_{\mathbb{R}^n} |f(x-t)-f(x)|^p dx\right)^{\frac{1}{p}} = \left(\int_{\mathbb{R}^n} |f(y)-f(y+t)|^p dy\right)^{\frac{1}{p}}=w_pf(-t).
\end{equation}
Continuamos observando que, en efecto, $0 \leq w_pf(t)$. Para la otra acotación, tomando $t \in \mathbb{R}^n$:
\begin{equation}
     w_pf(t) =||\tau_tf-f||_p \leq 2 ||f||_p.
\end{equation}
A continuación, probamos que $w_pf$ es uniformemente continua. 
Lo que haremos será utilizar el Teorema \ref{teo:sc}.

\noindent Dado $\epsilon \in \mathbb{R}^+$, existe una función $g$ continua con soporte compacto tal que $||f-g||_p < \frac{\epsilon}{3}$.

\noindent Como la función $g$ es continua con soporte compacto, sabemos que existe $R>0$, tal que $\{x \in \mathbb{R}^n : g(x) \neq 0\} \subset B_{||\cdot||_\infty}(0,R)$. Además, al ser g uniformemente continua, existe $0 < \delta <1$ tal que
\begin{equation}\label{eq:acot}
    x,y \in \mathbb{R}^n, |x-y| < \delta \implies |g(x)-g(y)| < \frac{\epsilon}{3M^{\frac{1}{p}}}\,,
\end{equation}
con $M = \mu({B_{||\cdot||_\infty}(0,R+1)})$ donde $\mu$ es la medida de Lebesgue.

\noindent Para cada $s,t \in \mathbb{R}^n$, tenemos que
\begin{equation}
    |w_pf(s)-w_pf(t)| \leq ||(\tau_sf-f)-(\tau_tf-f)||_p = ||\tau_sf-\tau_tf||_p.
\end{equation}
Usando la función $g$ podemos escribir
\begin{align}
||\tau_sf-\tau_tf||_p &= ||\tau_sf-\tau_tf +g -g + \tau_sg -\tau_sg +\tau_tg -\tau_tf || \\
&= ||\tau_s(f-g)+(\tau_sg-\tau_tg) - \tau_t (f-g)||_p 
\leq 2 ||f-g||_p + ||\tau_sg-\tau_tg||_p.
\end{align}
Tratamos entonces de acotar el segundo término. Tomamos $|s-t| < \delta $. Usando~\eqref{eq:acot} y teniendo en cuenta que $g(x-(s-t))=g(x) = 0 \,\, \forall x \notin B_{||\cdot||_\infty}(0,R+1)$, se tiene que 
\begin{equation}
    ||\tau_sg-\tau_tg||_p = \left( \int_{\mathbb{R}^n}|g(x-(s-t))-g(x)|^{p} \, dx\right)^{\frac{1}{p}} \leq \left( \int_{B_{||\cdot||_\infty}(0,R+1)}\left|\frac{\epsilon}{3M^{\frac{1}{p}}}\right|^p \, dx\right)^{\frac{1}{p}} \leq \frac{\epsilon}{3}.
\end{equation}

\noindent Deducimos finalmente que 
\begin{equation}
    |w_pf(s)-w_pf(t)|\leq 2 ||f-g||_p + ||\tau_sg-\tau_tg||_p \leq \epsilon.
\end{equation}


\noindent Luego $w_pf$ es uniformemente continua en $\mathbb{R}^n$. Si tomamos  $\underset{\substack{t \rightarrow 0}}{\lim}w_pf(t)=w_pf(0)=0$ obtenemos la última observación.








    
\end{proof}

